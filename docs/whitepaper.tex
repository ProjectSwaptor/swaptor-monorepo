\documentclass[12pt]{article}
\usepackage{fancyhdr}

\setlength{\oddsidemargin}{27mm}
\setlength{\evensidemargin}{27mm}
\setlength{\hoffset}{-1in}

\setlength{\topmargin}{27mm}
\setlength{\voffset}{-1in}
\setlength{\headheight}{0pt}
\setlength{\headsep}{0pt}

\setlength{\textheight}{235mm}
\setlength{\textwidth}{155mm}

%\pagestyle{empty}
\pagestyle{plain}

\renewcommand{\thefootnote}{\fnsymbol{footnote}}
\renewcommand{\labelitemi}{$\diamond$}

\begin{document}
\baselineskip 12pt

\pagestyle{fancy}
\fancyhf{}
\rhead{Current version: 1.0}

\begin{center}
  \textbf{\Large Swaptor Whitepaper} \\

  \vspace{1.5cc}
  { \sc Marko Ivanković$^{1}$}\\

  \vspace{0.3 cm}

  {\small $^{1}$Berry Block, marko@berryblock.io}
\end{center}
\vspace{1.5cc}

\begin{abstract}
  \noindent  Peer-to-peer (P2P) swaps on blockchain have the potential to greatly benefit users by eliminating the need for intermediaries in the exchange of assets. However, the use of P2P swaps on blockchain also presents several challenges, including trust issues that must be addressed in order to ensure their success.  Since there is no intermediary to oversee the exchange of assets, users must rely on the trustworthiness of the other party to the transaction. In the absence of a trusted third party, it is difficult to verify the authenticity and quality of the assets being exchanged, which can lead to disputes and losses for users.
  \\ \indent Furthermore, P2P swaps on blockchain are subject to potential security risks, such as hacking and fraud. Since the transactions are conducted directly between users, there is a greater risk of malicious actors attempting to exploit vulnerabilities in the system. This risk is exacerbated by the fact that blockchain transactions are irreversible, meaning that users have no recourse if their assets are stolen or lost.
  \\ \indent In this paper we describe Swaptor, a decentralized P2P exchange dapp which aims to eliminate problems mentioned above.

  \vspace{0.95cc}
\end{abstract}

\newpage

\tableofcontents

\newpage

\section{Architectural Overview} \label{form}
\indent As most of modern dapps, Swaptor's architecture is a mix of on-chain and off-chain components:
\begin{itemize}
  \item \textbf{On-chain}: Smart contracts
  \item \textbf{Off-chain}: Backend, Frontend and Database
\end{itemize}

\subsection{On-chain Architecture}
\indent Smart contract architecture is minimal, there is only a single contract called Swaptor.
Its main purpose is to verify digital signatures and execute the swaps on successful verification. 
What is important to add is that Swaptor contract never holds assets that are meant to be swapped.


\subsection{Off-chain Architecture}

\section{Security} \label{subm}
\dots

\section{Tokenomics}
\dots

\section{Roadmap}
\dots

\begin{thebibliography}{9}
  \bibitem{ex}
  Author(s), ``title,'' journal/proceedings info, page numbers,
  month \& year.
\end{thebibliography}

\end{document}